%%%%%%%%%%%%%%%%%%%%%%%%%%%%%%%%%%%%%%%%%
% Medium Length Professional CV
% LaTeX Template
% Version 2.0 (8/5/13)
%
% This template has been downloaded from:
% http://www.LaTeXTemplates.com
%
% Original author:
% Trey Hunner (http://www.treyhunner.com/)
%
% Important note:
% This template requires the resume.cls file to be in the same directory as the
% .tex file. The resume.cls file provides the resume style used for structuring the
% document.
%
%%%%%%%%%%%%%%%%%%%%%%%%%%%%%%%%%%%%%%%%%

%----------------------------------------------------------------------------------------
%	PACKAGES AND OTHER DOCUMENT CONFIGURATIONS
%----------------------------------------------------------------------------------------

\documentclass{resume} % Use the custom resume.cls style

\usepackage[left=0.75in,top=0.6in,right=0.75in,bottom=0.6in]{geometry} % Document margins
\usepackage{hyperref}
\usepackage{color}


\name{Jiaqi Yan} % Your name
\address{3117 S Lowe Ave \\ Chicago, IL 60616} % Your address
\address{(312)~$\cdot$~888~$\cdot$~1642 \\ littlepretty881203@gmail.com} % Your phone number and email
\address{{\href{https://github.com/littlepretty}{Github: littlepretty}} \\ Website: \href{http://mypages.iit.edu/~jyan31}{\textit{http://mypages.iit.edu/~jyan31}}}

\begin{document}
\lastupdated

%----------------------------------------------------------------------------------------
%	EDUCATION SECTION
%----------------------------------------------------------------------------------------

\begin{rSection}{Education}

{\bf Illinois Institute of Technology} \hfill {\em Expected Graduation May 2019} \\ 
Ph.D in Computer Science \\
Advisor: \href{http://cs.iit.edu/~djin/index.html}{Dr. Dong (Kevin) Jin} \\
Interests: Deep/Machine Learning for Network Security, Software-defined Networking, Container-based Network Virtualization and Emulation \\
Overall GPA: 4.0/4.0

{\bf Beijing Normal University} \hfill{\em Sept 2011 -- July 2014} \\
M.E. in Communication Systems\\
Advisor: \href{http://cist.bnu.edu.cn/xygk/szdw/zgj/552.html}{Dr. Zuying Luo} \\
Thesis: \emph{On Temperature and Power Management for VLSI} \\
Overall GPA: 3.85/4.0

{\bf Beijing Normal University} \hfill{\em Sept 2007 -- July 2011} \\
B.S. in Electronic Engineering\\
Thesis: \emph{On A Novel Prediction Model of Network Traffic} \\
Overall GPA: 3.47/4.0

\end{rSection}

%----------------------------------------------------------------------------------------
%	WORK EXPERIENCE SECTION
%----------------------------------------------------------------------------------------

\begin{rSection}{Experience}

\begin{rSubsection}{Nokia Bell Labs}{June 2017 - August 2017}{Summer Research Intern}{Murray Hill, NJ}
\item Works with \href{https://www.bell-labs.com/usr/sameer.sharma}{Sameer Sharama} and \href{https://www.bell-labs.com/usr/bilgehan.erman}{Bilgehan Erman} on the \textbf{Autonomic Mobile Network Function Migration} project
\item Study and propose the design of \textbf{stateless} Virtualized Network Function(VNF) to break the tight coupling of data and processing, along with its migration mechanism
\item Design and build the VNF migration framework in the SDN- and NFV-enabled 5G mobile network environment
\item Implement an ONOS traffic monitoring and steering network function(NF) in both stateful and stateless fashion, and experimentally demonstrate their migration mechanisms between edge and core clouds
\end{rSubsection}

%------------------------------------------------

\begin{rSubsection}{Fermi Accelerator National Laboratory}{July 2016 - Aug 2016}{Research Visitor}{Batavia, IL}
\item Works with \href{http://home.fnal.gov/~timm/}{Steven Timm} on the \textbf{HEPCloud Infrastructure} project
\item Dockerize high energy particle computation applications like MicroBoone and CMS to port to NERSC HPC systems
\item Document dockerization and job submission workflow on Edison \& Cori @ NERSC
\item Explore the possibility of utilizing Google Container Engine and managing dockers orchestration using Kubernetes
\end{rSubsection}

%------------------------------------------------
%\begin{rSubsection}{AsiaInfo Linkage}{July 2013 - August 2013}{iOS Developer Intern}{Beijing, China}
%\item Accelerate backtracking code from seconds to milliseconds for game-solving strategy 
%\item Design the RESTful communication interface between mobile end and server end
%\item Implement the audio/video processing component of an application
%\end{rSubsection}

\end{rSection}

%----------------------------------------------------------------------------------------
%	RESEARCH SECTION
%----------------------------------------------------------------------------------------

\begin{rSection}{Research}

\begin{rSubsection}{\textbf{Deep Learning} for Network Intrusion Detection System}{2016 - Present}{PhD Research Project}{Illinois Institute of Technology}
\item Training deep Multilayer Perceptron network with empirically efficient techniques, such as SGD and dropout
\item Learn useful, hierarchical features unsupervisedly with generative models such as \textbf{stacked Restricted Boltzmann Machines} or \textbf{stacked Sparse/Denoise Autoencoders}, visualizing using \textbf{t-SNE}
\item Synthesize new attacking traffics using various \textbf{Generative Adversarial Networks}
\item Compare prediction precision and recall of my \textbf{TensorFlow}-based implementations with traditional ML approaches, e.g., SVM, decision trees
\end{rSubsection}

%------------------------------------------------
\begin{rSubsection}{Virtual Time System for Linux Container}{2014 - 2016}{PhD Research Project}{Illinois Institute of Technology}
\item Design and implement \textbf{Virtual Time System} in \textbf{Linux kernel}; features includes time dilation and time freeze for Linux container
\item Sent patch to Linux timekeeping's maintainer \textbf{John Stultz} as RFC
\item Adding virtualized clock to \textbf{Mininet}, the most popular \textbf{open source} SDN emulator, so that it can emulate 10 GB/s level link or x10 network devices.
\item Integrate VTS with \textbf{DSSnet}, our hybrid modeling platform combining electric power distribution simulation with software-defined networking emulation
\end{rSubsection}

%------------------------------------------------
\begin{rSubsection}{Dynamic Temperature and Power Management for MPSoCs}{2012 - 2013}{Master Student Project}{Beijing Normal University}
\item Design and implement faster temperature analysis \textbf{open source} software for temperature analysis in Multi-Processor System on Chips
\end{rSubsection}

\end{rSection}

%----------------------------------------------------------------------------------------
%	PUBLICATION SECTION
%----------------------------------------------------------------------------------------

\begin{rSection}{Publications}

%------------------------------------------------
{\bf Simulation of a Software-defined Network as One Big Switch} \hfill {\em PADS 2017}\\ 
Jiaqi Yan, Xin Liu, Dong Jin \\
2017 ACM SIGSIM Conference on Principles of Advanced Discrete Simulation, NTU, Singapore

%------------------------------------------------
{\bf DSSnet: A Smart Grid Modeling Platform Combining Electrical Power Distribution System
    Simulation and Software Defined Networking Emulation} \hfill {\em PADS 2016}\\ 
Christopher Hannon, Jiaqi Yan,Dong Jin \hfill {\color{red}Best Paper Nominee} \\
2016 ACM SIGSIM Conference on Principles of Advanced Discrete Simulation, Banff, Alberta, Canada

%------------------------------------------------
{\bf VT-Mininet: Virtual-time-enabled Mininet for Scalable and Accurate Software-Defined Network Emulation} \hfill {\em SOSR 2015}\\ 
Jiaqi Yan, Dong Jin \hfill {\color{red}Acceptance Rate 19.7\%} \\
2015 ACM SIGCOMM Symposium on SDN Research, San Jose, CA, USA

%------------------------------------------------
{\bf A Virtual Time System for Linux-container-based Emulation of Software Defined Networks} \hfill {\em PADS 2015}\\
Jiaqi Yan, Dong Jin \hfill {\color{red}Best Paper Nominee} \\
2015 ACM SIGSIM Conference on Principles of Advanced Discrete Simulation, London, UK

\end{rSection}

%----------------------------------------------------------------------------------------
%	AWARDS SECTION
%----------------------------------------------------------------------------------------

\begin{rSection}{HONORS}

%------------------------------------------------
{\bf Session Chair at 2017 ACM SIGSIM Conference on Principles of Advanced Discrete Simulation} \hfill {\em 2017}\\
Chair of the ``Simulation Applications" Session

%------------------------------------------------
{\bf Teaching Assistant: CS458 Cyber-Security} \hfill {\em CS Department, IIT}

%------------------------------------------------
{\bf {\color{red}Best} Poster Award} \hfill {\em Workshop on Science of Security through Software-Defined Networking}\\
Towards A Secure and Resilient Industrial Control System with Software-Defined Networking \\
Dong Jin, Jiaqi Yan, Xin Liu, Christopher Hannon

%------------------------------------------------
{\bf {\color{red}First Place} in Student Poster Session} \hfill {\em 2016 IIT Research Day} \\
DSSnet: A Smart Grid Modeling Platform Combining Electrical Power Distribution System Simulation and Software Defined Networking Emulation \\
Christopher Hannon, Jiaqi Yan, Dong Jin

{\bf Graduate Student Scholarship} \hfill {\em 2014 Beijing Normal University} \\
Awarded to the \textbf{top 5} student in their final year of graduate degree

{\bf Postgraduate Student Scholarship} \hfill {\em 2008 Beijing Normal University} \\
{Awarded to the \textbf{top 10} student in their first year of Bachelors degree}

\end{rSection}

%----------------------------------------------------------------------------------------
%	TECHNICAL STRENGTHS SECTION
%----------------------------------------------------------------------------------------

\begin{rSection}{Technical Strengths}

\begin{tabular}{ @{} >{\bfseries}l @{\hspace{6ex}} l }
Computer Languages & C/C++, Python, Shell Script, \LaTeX, Object-C, Swift \\
Platforms & Linux Kernel Development, TensorFlow, Keras, ONOS, iOS \\ 
Protocols & OpenFlow, TCP/IP, REST, JSON \\
Databases & MySQL, Microsoft SQL \\
Tools & Vim, Git, Xcode, Matlab
\end{tabular}

\end{rSection}

%----------------------------------------------------------------------------------------
%	EXAMPLE SECTION
%----------------------------------------------------------------------------------------

%\begin{rSection}{Section Name}

%Section content\ldots

%\end{rSection}

%----------------------------------------------------------------------------------------

\end{document}
